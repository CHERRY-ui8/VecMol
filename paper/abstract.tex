\begin{abstract}
Generative modeling of three-dimensional molecular structures remains challenging in drug discovery and materials science, particularly due to limitations in scalability and representational expressivity of existing approaches. We introduce a representation that models molecules as continuous vector fields pointing toward atomic centers, enabling arbitrary-resolution reconstruction and efficient generation. Our method decouples representation learning from generation via a two-stage framework: a neural field autoencoder learns continuous molecular fields with E(n)-equivariant graph neural networks, followed by a diffusion model operating in a fixed-dimensional latent space. A central technical contribution is a composite field definition that explicitly separates direction selection from magnitude control through a softmax--tanh formulation, resulting in more stable optimization and more accurate molecular reconstruction. Experiments on QM9 and GEOM-drugs datasets demonstrate competitive generation quality compared to state-of-the-art methods, achieving comparable performance in molecular stability, validity, and distributional coverage while maintaining favorable trade-offs between generation quality and structural stability for larger drug-sized molecules.
\end{abstract}
