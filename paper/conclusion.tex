\section{Conclusion}
\label{sec:conclusion}

We have presented \textsc{VecMol}, a novel approach for unconditional 3D molecule generation that represents molecules as continuous vector-valued neural fields. By encoding directional gradient information rather than scalar densities, our method enables direct gradient-based reconstruction while maintaining geometric equivariance through E($n$)-equivariant graph neural networks. The proposed Softmax--Tanh field formulation addresses key challenges in field-to-molecule conversion, providing bounded, well-conditioned gradients that facilitate stable learning and accurate reconstruction. Our experiments demonstrate that diffusion in the latent vector-field space produces chemically valid and stable molecules, achieving competitive performance on QM9 and GEOM-drugs datasets, with particular advantages in molecular stability and geometric consistency on larger drug-like molecules.

While our approach offers improved scalability compared to point-cloud methods and better representational capacity than voxel-based approaches, limitations remain. The method's performance on small molecules, where point-cloud diffusion models excel, suggests room for improvement in distributional coverage. Future work could explore more sophisticated field definitions, investigate conditional generation for structure-based drug design, and extend the framework to handle larger molecular systems and macrocycles. The vector-field representation opens new avenues for molecular modeling, bridging the gap between continuous field representations and discrete atomic structures in a principled, scalable manner.
